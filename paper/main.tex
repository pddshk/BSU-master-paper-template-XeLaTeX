% !TeX spellcheck = ru_RU
% !TeX TXS-program:bibliography = txs:///bibtex/{} output/main.aux
% !TeX TXS-program:prepare = txs:///xelatex/[-synctex=0 -output-directory=output -no-pdf -shell-escape]
% !TeX TXS-program:recompile-bibliography = txs:///prepare | txs:///bibliography | txs:///prepare
% !TeX TXS-program:compile = txs:///xelatex/[-synctex=15 -output-directory=output -shell-escape] | mv output/%.pdf ./ | mv output/%.log ./

\documentclass{mpaper}

\usepackage{tikz}
\usepackage[outputdir=output]{minted}

\setmainfont{PTSerif} % PTSerif.fontspec

%\title{Центральные простые алгебры}
%\author{ДЫДЫШКО Павел Александрович}
%\faculty{механико--математический факультет}
%\department{высшей алгебры и защиты информации}
%\speciality{1-31 80 03 <<математика и компьютерные науки>>}
%\supervisor{Тихонов Сергей Викторович\\доцент, кандидат физ.--мат. наук}
%\headofdepartment{Беняш--Кривец, Валерий Вацлавович\\профессор, доктор физ.--мат. наук}
%\setyear{2024}

\renewcommand{\LaTeXe}{\LaTeX\kern.15em2$_{\textstyle\varepsilon}$}
\setlength{\parskip}{0.3\baselineskip}
\begin{document}
    \maketitle
    \setcounter{tocdepth}{3}
    \setcounter{page}{2}
    \tableofcontents

    \newpage
    \noindent baselineskip is \the\baselineskip\\
    font size is \makeatletter\f@size\makeatother pt\\
    large font size is {\normalsizerrr \makeatletter\f@size\makeatother} pt\\
    parindent is \the\parindent\ which is 1.25cm

    Прежде чем вообще верстать документ, следует ознакомиться с \href{http://anorien.csc.warwick.ac.uk/mirrors/CTAN/info/l2tabu/english/l2tabuen.pdf}{\selectlanguage{english}\bfseries An essential guide to \LaTeXe\ usage}

    В документе дохренища оверфуллов из-за дебильных требований к полям и размерам шрифтов (излагайте свои мысли красиво и аккуратно, чтобы они нормально переносились или же попробуйте юзать так рекомендованный таймс нью роман, мб логи будут почище) ни в коем случае не юзайте \verb|\sloppy| или я приду к вам ночью и отрежу жопу

    Обращаю также внимание, что обычные дефисы ломают переносы слов, таким образом, например пишущееся черезе дефис механико-математический в этой строчке вызывает оверфулл, ибо не может быть нормально перенесено. Это происходит потому, что \TeX\ не переносит слова, в которых переносы заданы явно. Во избежание таких недоразуменый русскоязычный \verb|babel| вводит следующий знак переноса: механико\verb|"=|математический механико"=математический

%    \chapter*{\large РЕФЕРАТ}  
\addcontentsline{toc}{chapter}{РЕФЕРАТ}



\newpage
\chapter*{\large РЭФЕРАТ} 



\newpage
\chapter*{\large ABSTRACT} 



\newpage

    \chapter*{Перечень условных обозначений и сокращений}
\addcontentsline{toc}{chapter}{Перечень условных обозначений и сокращений}
В случае если в диссертации используются специфическая терминология, малораспространенные сокращения, аббревиатуры, условные обозначения и тому подобное, их объединяют в <<Перечень условных обозначений и сокращений>>. Условные обозначения располагаются в алфавитном порядке в виде нумерованной колонки, справа от них дается их расшифровка.
    
    % !TeX root = main
\chapter*{Введение}
\addcontentsline{toc}{chapter}{Введение}

Во <<Введении>> (до 3 страниц) обосновывается актуальность темы, её значение, выбор направления исследования, необходимость проведения исследований по данной теме для решения конкретной проблемы (задачи), развития конкретных направлений в соответствующих областях науки, отраслях экономики.

    \chapter{Общая характеристика работы}
<<Общая характеристика работы>> содержит:

перечень ключевых слов;

цель, задачи, объект и предмет исследования;

формулировку полученных результатов и их новизну;

сведения о структуре магистерской диссертации.

Перечень ключевых слов характеризует основное содержание магистерской диссертации и включает 10-15 слов в именительном падеже, написанных через запятую в строку прописными буквами.

При описании структуры магистерской диссертации кратко излагается и поясняется логика ее построения, в том числе полный объем работы в страницах; объем, занимаемый иллюстрациями, таблицами, приложениями (с указанием их количества); количество использованных источников (включая собственные публикации магистранта).

При изложении текста раздела «Общая характеристика работы»
следует употреблять синтаксические конструкции, свойственные языку
научных документов, использовать стандартизованную терминологию,
избегать сложных грамматических оборотов, малораспространенных
терминов и символов.

«Общая характеристика работы» выполняется на трех языках: русском, белорусском и одном из иностранных языков по выбору студента. Иностранные граждане могут выполнять «Общую характеристику работы» на двух языках: русском и иностранном.

Для <<Общей характеристики работы>> оптимальный объем текста составляет 1500--2000 печатных знаков (примерно одна страница).

Ссылка на \cref{fig:1}
\begin{figure}
    \centering
    \begin{tikzpicture}
        \draw (0,0) -- (1,1) -- (0,1) -- (1,0) -- cycle;
    \end{tikzpicture}
    \caption{Какая-то хрень}\label{fig:1}
\end{figure}

\section{фф}

\section{фф}

\section{фф}

фф

\section{фф}

фф

\section{фф}

фф

\chapter{Основная часть}
Основная часть материала диссертации излагается в главах, в которых приводятся:

аналитический обзор литературы по теме~--– анализ работ, выполненных ранее отечественными и зарубежными исследователями, описание имеющихся подходов к исследованию проблемы, оценка степени изученности вопроса, формулирование проблемы, которая остается неразрешенной;

описание объектов исследования и используемых при проведении исследования методов, оборудования~--- характеристика основных подходов к решению поставленных задач, используемых теоретических и (или) экспериментальных методов и обоснование целесообразности их использования;

изложение выполненных в работе теоретических и (или) экспериментальных исследований.

Распределение основного материала магистерской диссертации по главам (разделам) и по параграфам (подразделам) определяются магистрантом. Весь порядок изложения в диссертации должен быть подчинен цели исследования, сформулированной автором. Дробление материала диссертации на главы, разделы, подразделы, а также их последовательность должны быть логически оправданными.

Каждую главу диссертации следует завершать краткими выводами.
    % !TeX root = main
\chapter*{Заключение}
\addcontentsline{toc}{chapter}{Заключение}
В «Заключении» формулируются основные результаты исследования и практические рекомендации по их использованию.

Выводы формулируются по пунктам и излагаются последовательно по каждому разделу магистерской диссертации. Выводы должны быть конкретными и обоснованными, вытекать из содержания магистерской диссертации.

«Список использованных источников» содержит перечень источников информации, на которые в диссертации приводятся ссылки, в том числе нормативные правовые акты по теме исследования, учебники, монографии и статьи отечественных и зарубежных авторов, в том числе на иностранных языках, ресурсы удаленного доступа, статистические материалы и др.

В <<Список использованных источников>> также включаются публикации магистранта по теме диссертации.~\cite{bib-prikaz} \cite{bibtex} Библиографию добавлять в \texttt{refs.bib}
    % !TeX root = main
\bibliographystyle{ugost2008}
\bibliography{refs}
\addcontentsline{toc}{chapter}{Список использованных источников}
    % !TeX root = main
\appendix
\chapter{Информация о приложениях}
В разделе <<Приложения>> размещается вспомогательный материал, позволяющий более полно раскрыть содержание и результаты исследования, оценить их научную и практическую значимость, в том числе:

промежуточные математические доказательства, формулы и расчеты, оценки погрешности измерений;

исходные тексты компьютерных программ и краткое их описание;

таблицы и иллюстрации вспомогательного характера;

документы или их копии, которые подтверждают научное и (или) практическое применение результатов исследований или рекомендации по их использованию: акты (справки) о промышленных испытаниях, производственной проверке законченных научных разработок, применении полученных результатов и другое.

Число приложений определяется автором магистерской диссертации.

\chapter{Код на расте}

\begin{minted}{rust}
fn main() {
    println("Hello world");
}
\end{minted}

\end{document}