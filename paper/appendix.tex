% \titleformat{\section}[block]
%   {\large\bfseries\centering}
%   {\thesection\ }{}{}
% \chapter*{ПРИЛОЖЕНИЕ А}
% \addcontentsline{toc}{chapter}{ПРИЛОЖЕНИЕ А}
% \section*{\centering something}
% \begin{footnotesize}
% \begin{lstlisting}
        

    
% \end{lstlisting}
% \end{footnotesize}

\appendix
\chapter{Информация о приложениях}
В разделе <<Приложения>> размещается вспомогательный материал, позволяющий более полно раскрыть содержание и результаты исследования, оценить их научную и практическую значимость, в том числе:

промежуточные математические доказательства, формулы и расчеты, оценки погрешности измерений;

исходные тексты компьютерных программ и краткое их описание;

таблицы и иллюстрации вспомогательного характера;

документы или их копии, которые подтверждают научное и (или) практическое применение результатов исследований или рекомендации по их использованию: акты (справки) о промышленных испытаниях, производственной проверке законченных научных разработок, применении полученных результатов и другое.

Число приложений определяется автором магистерской диссертации.

\begin{minted}{rust}
  fn main() {
    println("Hello world");
  }
\end{minted}